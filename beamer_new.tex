<<<<<<< HEAD

\documentclass[hyperref={pdfpagelabels=false}]{beamer}
% Die Hyperref Option hyperref={pdfpagelabels=false} verhindert die Warnung:
% Package hyperref Warning: Option `pdfpagelabels' is turned off
% (hyperref)                because \thepage is undefined. 
% Hyperref stopped early 
%
% \input{longheader.tex}
% \input{global.tex}

\usepackage{lmodern}
% Das Paket lmodern erspart die folgenden Warnungen:
% LaTeX Font Warning: Font shape `OT1/cmss/m/n' in size <4> not available
% (Font)              size <5> substituted on input line 22.
% LaTeX Font Warning: Size substitutions with differences
% (Font)              up to 1.0pt have occurred.
%

% % % % % % % % % % % % % % % % % % % % % % % % % % % % % % % % % % % % % % % % % % % %
\usepackage{siunitx}
\sisetup{load-configurations=abbreviations}
\sisetup{
	%locale=DE,
	seperr=true,                    % Fehler anzeigen
	tightpm,                        % Abstand zwischen Fehler verringern
	tophrase={{\text{ bis }}},
	fraction=nice,
	per-mode=fraction,
	free-standing-units=true,
	space-before-unit=true,
	use-xspace=true,
	group-separator={{\text{~}}},
	list-final-separator={{\text{ und }}}
}
\usepackage{natbib}
\usepackage[labelformat=empty]{caption}
\usepackage{movie15}
\usepackage{xcolor,colortbl}
\usepackage{slashed}
\usepackage{amsfonts}
\usepackage{amssymb}
\usepackage{amsmath}
\usepackage{amscd}
\usepackage{amstext}
\usepackage[ngerman,german]{babel, varioref}
\usepackage[T1]{fontenc}
\usepackage[utf8]{inputenc}
\usepackage{xfrac}
\usepackage{booktabs}

% % % % % % % % % % % % % % % % % % % % % % % % % % % % % % % % % % % % % % % % % % % % % % % % %
% Wenn \titel{\ldots} \author{\ldots} erst nach \begin{document} kommen,
% kommt folgende Warnung:
% Package hyperref Warning: Option `pdfauthor' has already been used,
% (hyperref) ... 
% Daher steht es hier vor \begin{document}

\title{Hamilton-Formalismus in der Beschleunigerphysik}  
\institute{
Technische Universit\"at Dortmund}
\author{Sonja Bartkowski, Dimitrios Skodras} 
\date{11.06.2015} 

% zusaetzlich ist das usepackage{beamerthemeshadow} eingebunden 
\usepackage{beamerthemeshadow}


%  \beamersetuncovermixins{\opaqueness<1>{25}}{\opaqueness<2->{15}}
%  sorgt dafuer das die Elemente die erst noch (zukuenftig) kommen 
%  nur schwach angedeutet erscheinen 
\beamersetuncovermixins{\opaqueness<1>{25}}{\opaqueness<2->{15}}
% klappt auch bei Tabellen, wenn teTeX verwendet wird\ldots

\beamertemplatenavigationsymbolsempty

\begin{document}

\setbeamertemplate{footline}
{%
  \leavevmode%
 \begin{beamercolorbox}%
    [wd=.5\paperwidth,ht=2.5ex,dp=1.125ex,leftskip=.3cm,rightskip=.3cm]%
    {author in head/foot}%
    \usebeamerfont{author in head/foot}%
    \hfill\insertshortauthor
  \end{beamercolorbox}%
  \begin{beamercolorbox}%
    [wd=.5\paperwidth,ht=2.5ex,dp=1.125ex,leftskip=.3cm ,rightskip=.3cm]%
    {title in head/foot}%
    \usebeamerfont{title in head/foot}%
    \insertshorttitle\hfill\insertframenumber{}
  \end{beamercolorbox}%
}%

\setbeamertemplate{caption}{\raggedright\insertcaption\par}
\captionsetup[figure]{font=small,skip=0pt}
\begin{frame}
\titlepage
\end{frame} 

\begin{frame}
\frametitle{Gliederung}
\tableofcontents
\end{frame} 

\newcommand{\tmotiv}{Grundlagen}
\section{\tmotiv}
% 
 \begin{frame}
   \frametitle{Standardmodell}
   \pause
   \begin{minipage}[h]{0.48\textwidth}
    \begin{itemize}
    \visible<2->{\item Teilcheninhalt:}
    \begin{itemize}
     \visible<3->{\item Leptonen: $e^+$, $\mu^+$, $\nu$}
     \visible<4->{\item Quarks: $\bar u$, $\bar d$, $s$, $c$}
     \visible<5->{\item Vektorbosonen: $W^+$, $g$}
     \visible<6->{\item (Skalarbosonen: $H$, $\pi$)}
    \end{itemize}  
    \visible<2->{\item Fundamentale Wechselwirkungen:} 
    \begin{itemize}
     \visible<7->{\item starke Wechselwirkung (QCD)}
     \visible<8->{\item elektroschwache Wechselwirkung (GSW-Theorie)}
    \end{itemize}  
   \end{itemize}
   \end{minipage}

 \end{frame}
 
 \begin{frame}
  \frametitle{Feynmangraph}
   \setcounter{framenumber}{4}
 \begin{minipage}[h]{0.38\textwidth}
  \begin{enumerate}
   \item ruhendes $D$-Meson
 \end{enumerate}
 \end{minipage}
 \begin{minipage}[h]{0.58\textwidth}

 \end{minipage}
 \end{frame}
 
 \begin{frame}
  \frametitle{Feynmangraph}
 \setcounter{framenumber}{4}
 \begin{minipage}[h]{0.38\textwidth}
  \begin{enumerate}
   \item ruhendes $D$-Meson
   \item propagiert in $t$
  \end{enumerate}
 \end{minipage}
 \begin{minipage}[h]{0.58\textwidth}
 \end{minipage}
 \end{frame}
 
 \begin{frame}
  \frametitle{Feynmangraph}
  \setcounter{framenumber}{4}
 \begin{minipage}[h]{0.38\textwidth}
  \begin{enumerate}
   \item ruhendes $D$-Meson
   \item propagiert in $t$.
   \item $c$ wandelt unter Abstrahlung von $W^+$ in $s$ 
  \end{enumerate}
 \end{minipage}
 \begin{minipage}[h]{0.58\textwidth}

 \end{minipage}
 \end{frame}
 
 \begin{frame}
  \frametitle{Feynmangraph}
  \setcounter{framenumber}{4}
 \begin{minipage}[h]{0.38\textwidth}
  \begin{enumerate}
   \item ruhendes $D$-Meson
   \item propagiert in $t$.
   \item $c$ wandelt unter Abstrahlung von $W^+$ in $s$
   \item $W^+$ zerstrahlt in Leptonpaar $l^+$, $\nu_l$
  \end{enumerate}
 \end{minipage}
 \begin{minipage}[h]{0.58\textwidth}

 \end{minipage}
 \end{frame}
 
 \begin{frame}
 \frametitle{\"Uberblick}
 \pause
  \begin{minipage}[h]{0.62\textwidth}
  \begin{itemize}
   \visible<2->{\item Fermis Goldene Regel für Zerfälle:}\visible<3->{ $\underbrace{\mathrm{d}\Gamma}_{\text{Breite}} = \frac{1}{2m_D} \underbrace{\mathrm{d}\Phi}_{\text{Phasenraum}}\, \cdot \,{\underbrace{|M|}_{\text{Amplitude}}}^2$}
   \visible<2->{\item Teilchenstr\"ome}
   \begin{itemize}
    \visible<4->{\item [$\circ$] relativistischer Dirac-Strom}
    \visible<5->{\item [$\circ$] kurze Reichweite von $W^+$ f\"ur geringe Energien \\$\rightarrow$ Beschreibung durch 4-Fermionen-Wechselwirkung}
   \end{itemize}
 
   \visible<2->{\item Starke WW zwischen $c$ und $\bar q_1$}
   \begin{itemize}
    \visible<6->{\item[$\circ$] erh\"alt Parit\"at $\mathcal{P}$}
    \visible<7->{\item[$\circ$] st\"orungsrechnerisch nicht erfassbar \\$\rightarrow$ Darstellung durch \textbf{Formfaktoren} $f$}
   \end{itemize}
  \end{itemize}
 \end{minipage}
 \visible<1->{\begin{minipage}[h]{0.34\textwidth}

 \end{minipage}}
 \end{frame}


\section{Geladene Teilchen im EM-Feld}
\begin{frame}
\tableofcontents[currentsection]
\end{frame}
\subsection{Relativistik}
 \begin{frame}
    \frametitle{Zerfallsbreite}
    \pause
    \begin{itemize}
     \visible<2->{\item Inverses der hier sehr kurzen Lebensdauer $\tau$}
     \visible<3->{\item Energiemessung führt wegen Energieunschärfe zu Verteilungen}
     \visible<4->{\item [$\rightarrow$] Breite der Verteilung $\Gamma$ kann gemessen werden}
     \end{itemize}
 \end{frame}

\begin{frame}
 \frametitle{Ergebnis der differentiellen Zerfallsbreite}
 Fermis Goldene Regel:\\
 \pause
 \begin{align*}
 \visible<2->{\mathrm{d} \Gamma(D\rightarrow Kl\nu) &= \frac{|M|^2}{2m_D} \mathrm{d}\Phi(K,\,l,\,\nu)} \nonumber\\
 \visible<3->{&= \frac{G_F^2 |V_{cs}|^2}{24\pi^3}|f_+(q^2)|^2|p_K|^3\mathrm{d}q^2}
 \label{eq_theoGamma}
 \end{align*}
 \vskip-1.5em
 \vspace{0.5cm}
 \visible<3->
 {\small{Fermikonstante $G_F$,\\ CKM-Element $V_{cs}$,\\ Formfaktor $f_+$,\\Kaonimpuls $p_K$,\\ Impuls\"ubertrag $q^2$}}
\end{frame}



\subsection{Transformation ins mitbewegete System}
\begin{frame}
\frametitle{Phasenraumvolumen}
 \begin{itemize}
  \visible<2->{\item Enthält kinematische Informationen (Energien, Impulse)}
  \visible<3->{\item Je mehr Endzustände existieren, umso größer ist $\Phi$}
  \visible<4->{\item Nicht vom Matrixelement unabhängig berechenbar, da es Viererimpulse enthält}
 \end{itemize}
 \visible<5->{Ein erster Ausdruck:
 \begin{align*}
  \mathrm{d} \Phi & = (2\pi)^4\frac{\mathrm{d}^3p_K}{2(2\pi)^3E_K} \frac{\mathrm{d}^3k_1}{2(2\pi)^3E_1} \frac{\mathrm{d}^3k_2}{2(2\pi)^3E_2}\delta^4(p_D-p_K-k_1-k_2)\nonumber
 \end{align*}}
\end{frame}

\subsection{Beispiele}
\begin{frame}
\frametitle{senkrechte Magnetfelder}
 \pause
 \begin{itemize}
  \visible<2->{\item Enthält dynamische Informationen (Wechselwirkungen)}
  \visible<3->{\item Beschreibt Übergang ähnlich Streuung von Startzustand $i$ zu Endzustand $f$}
  \visible<4->{\item Betragsquadrat $|M|^2$ kann als Wahrscheinlichkeit für Reaktion betrachtet werden}
 \end{itemize}
  \visible<5->{Ein erster Ausdruck:
  \begin{align*}
   M = \langle Kl\nu\,|\mathcal{H}|\,D\rangle
  \end{align*}}
\end{frame}

\begin{frame}
 \frametitle{kleine Winkel zur Sollbahn}
\end{frame}

\begin{frame}
 \frametitle{kleine Impulsabweichungen zum Sollimpuls}
\end{frame}

\section{Transformation auf Wirkungs-Winkel-Variable}
\subsection{Bedeutng der Wirkungs-Winkel-Variablen}
\begin{frame}
 \frametitle{Bedeutende Größen}
\end{frame}

\subsection{Beispiele}
\begin{frame}
 \frametitle{Beispiel: Gradientenfehler}
\end{frame}

\begin{frame}
 \frametitle{Beispiel: Sextupol}
\end{frame}

\section{Resonanzen}
\begin{frame}
 \frametitle{Beispiel: Sextupol}
\end{frame}

\begin{frame}
 \frametitle{Verhalten in Resonanznähe}
\end{frame}

\begin{frame}
 \frametitle{Fixpunkte}
\end{frame}

\section{Ausblick}
\begin{frame}
 \frametitle{Oktupol}
\end{frame}

\begin{frame}
 \frametitle{Kopplung}
\end{frame}

\begin{frame}
 \frametitle{Gegenwart vieler Nichtlinearitäten}
\end{frame}




\end{document}
=======

\documentclass[hyperref={pdfpagelabels=false}]{beamer}
% Die Hyperref Option hyperref={pdfpagelabels=false} verhindert die Warnung:
% Package hyperref Warning: Option `pdfpagelabels' is turned off
% (hyperref)                because \thepage is undefined. 
% Hyperref stopped early 
%
% \input{longheader.tex}
% \input{global.tex}

\usepackage{lmodern}
% Das Paket lmodern erspart die folgenden Warnungen:
% LaTeX Font Warning: Font shape `OT1/cmss/m/n' in size <4> not available
% (Font)              size <5> substituted on input line 22.
% LaTeX Font Warning: Size substitutions with differences
% (Font)              up to 1.0pt have occurred.
%

% % % % % % % % % % % % % % % % % % % % % % % % % % % % % % % % % % % % % % % % % % % %
\usepackage{siunitx}
\sisetup{load-configurations=abbreviations}
\sisetup{
	%locale=DE,
	seperr=true,                    % Fehler anzeigen
	tightpm,                        % Abstand zwischen Fehler verringern
	tophrase={{\text{ bis }}},
	fraction=nice,
	per-mode=fraction,
	free-standing-units=true,
	space-before-unit=true,
	use-xspace=true,
	group-separator={{\text{~}}},
	list-final-separator={{\text{ und }}}
}
\usepackage{natbib}
\usepackage[labelformat=empty]{caption}
\usepackage{movie15}
\usepackage{xcolor,colortbl}
\usepackage{slashed}
\usepackage{amsfonts}
\usepackage{amssymb}
\usepackage{amsmath}
\usepackage{amscd}
\usepackage{amstext}
\usepackage[ngerman,german]{babel, varioref}
\usepackage[T1]{fontenc}
\usepackage[utf8]{inputenc}
\usepackage{xfrac}
\usepackage{booktabs}

% % % % % % % % % % % % % % % % % % % % % % % % % % % % % % % % % % % % % % % % % % % % % % % % %
% Wenn \titel{\ldots} \author{\ldots} erst nach \begin{document} kommen,
% kommt folgende Warnung:
% Package hyperref Warning: Option `pdfauthor' has already been used,
% (hyperref) ... 
% Daher steht es hier vor \begin{document}

\title[Hamiltonformalismus]{Hamiltonformalismus in der Beschleunigerphysik}  
\institute{Beschleunigerphysik I \& II\\
Technische Universit\"at Dortmund}
\author{Dimitrios Skodras \and Sonja Bartkowski} 
\date{11.06.2015} 

% zusaetzlich ist das usepackage{beamerthemeshadow} eingebunden 
\usepackage{beamerthemeshadow}


%  \beamersetuncovermixins{\opaqueness<1>{25}}{\opaqueness<2->{15}}
%  sorgt dafuer das die Elemente die erst noch (zukuenftig) kommen 
%  nur schwach angedeutet erscheinen 
\beamersetuncovermixins{\opaqueness<1>{25}}{\opaqueness<2->{15}}
% klappt auch bei Tabellen, wenn teTeX verwendet wird\ldots

\beamertemplatenavigationsymbolsempty

\begin{document}

\setbeamertemplate{footline}
{%
  \leavevmode%
 \begin{beamercolorbox}%
    [wd=.5\paperwidth,ht=2.5ex,dp=1.125ex,leftskip=.3cm,rightskip=.3cm]%
    {author in head/foot}%
    \usebeamerfont{author in head/foot}%
    \hfill\insertshortauthor
  \end{beamercolorbox}%
  \begin{beamercolorbox}%
    [wd=.5\paperwidth,ht=2.5ex,dp=1.125ex,leftskip=.3cm ,rightskip=.3cm]%
    {title in head/foot}%
    \usebeamerfont{title in head/foot}%
    \insertshorttitle\hfill\insertframenumber{}
  \end{beamercolorbox}%
}%

\setbeamertemplate{caption}{\raggedright\insertcaption\par}
\captionsetup[figure]{font=small,skip=0pt}
\begin{frame}
\titlepage
\end{frame} 

\begin{frame}
\frametitle{Gliederung}
\tableofcontents
\end{frame} 


\section{Grundlagen}
 \begin{frame}
   \frametitle{Sonjas Teil}
   \pause
 \end{frame}
 

\section{Hamilton-Funktion für geladenen Teilchen im EM-Feld}
\begin{frame}
\frametitle{Dimitris Teil}
\tableofcontents[currentsection]
\end{frame}
\subsection{Gibt es hier Unterkapitel}
 

\section{Transformation auf Wirkungs-Winkel-Variable}
\begin{frame}
\frametitle{Sonjas Teil}
\tableofcontents[currentsection]
\end{frame}

\section{Kanonische Störungsrechnung}
\begin{frame}
\tableofcontents[currentsection]
\end{frame}
\subsection{Sextupol}
\begin{frame}
\frametitle{Sextupol - Sonjas Teil}
\end{frame}
\subsection*{Rest - Dimitris Teil?}
\begin{frame}
\frametitle{Rest - Dimitris Teil?}
 \framesubtitle{Motivation}
\end{frame}
\end{document}
>>>>>>> fb1dc7e1283ceaffe2a0f14f4c81cdfe63ef3826
