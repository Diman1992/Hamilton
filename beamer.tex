
\documentclass[hyperref={pdfpagelabels=false}]{beamer}
% Die Hyperref Option hyperref={pdfpagelabels=false} verhindert die Warnung:
% Package hyperref Warning: Option `pdfpagelabels' is turned off
% (hyperref)                because \thepage is undefined. 
% Hyperref stopped early 
%
% \input{longheader.tex}
% \input{global.tex}

\usepackage{lmodern}
% Das Paket lmodern erspart die folgenden Warnungen:
% LaTeX Font Warning: Font shape `OT1/cmss/m/n' in size <4> not available
% (Font)              size <5> substituted on input line 22.
% LaTeX Font Warning: Size substitutions with differences
% (Font)              up to 1.0pt have occurred.
%

% % % % % % % % % % % % % % % % % % % % % % % % % % % % % % % % % % % % % % % % % % % %
\usepackage{siunitx}
\sisetup{load-configurations=abbreviations}
\sisetup{
	%locale=DE,
	seperr=true,                    % Fehler anzeigen
	tightpm,                        % Abstand zwischen Fehler verringern
	tophrase={{\text{ bis }}},
	fraction=nice,
	per-mode=fraction,
	free-standing-units=true,
	space-before-unit=true,
	use-xspace=true,
	group-separator={{\text{~}}},
	list-final-separator={{\text{ und }}}
}
\usepackage{natbib}
\usepackage[labelformat=empty]{caption}
\usepackage{movie15}
\usepackage{xcolor,colortbl}
\usepackage{slashed}
\usepackage{amsfonts}
\usepackage{amssymb}
\usepackage{amsmath}
\usepackage{amscd}
\usepackage{amstext}
\usepackage[ngerman,german]{babel, varioref}
\usepackage[T1]{fontenc}
\usepackage[utf8]{inputenc}
\usepackage{xfrac}
\usepackage{booktabs}

% % % % % % % % % % % % % % % % % % % % % % % % % % % % % % % % % % % % % % % % % % % % % % % % %
% Wenn \titel{\ldots} \author{\ldots} erst nach \begin{document} kommen,
% kommt folgende Warnung:
% Package hyperref Warning: Option `pdfauthor' has already been used,
% (hyperref) ... 
% Daher steht es hier vor \begin{document}

\title[Formfaktoren von $D\rightarrow K l \nu$]{Formfaktoren des semileptonischen $D\rightarrow K l \nu$ Zerfalls}  
\institute{Lehrstuhl f\"ur Theoretische Physik IV\\
Technische Universit\"at Dortmund}
\author{Dimitrios Skodras} 
\date{03.09.2014} 

% zusaetzlich ist das usepackage{beamerthemeshadow} eingebunden 
\usepackage{beamerthemeshadow}


%  \beamersetuncovermixins{\opaqueness<1>{25}}{\opaqueness<2->{15}}
%  sorgt dafuer das die Elemente die erst noch (zukuenftig) kommen 
%  nur schwach angedeutet erscheinen 
\beamersetuncovermixins{\opaqueness<1>{25}}{\opaqueness<2->{15}}
% klappt auch bei Tabellen, wenn teTeX verwendet wird\ldots

\beamertemplatenavigationsymbolsempty

\begin{document}

\setbeamertemplate{footline}
{%
  \leavevmode%
 \begin{beamercolorbox}%
    [wd=.5\paperwidth,ht=2.5ex,dp=1.125ex,leftskip=.3cm,rightskip=.3cm]%
    {author in head/foot}%
    \usebeamerfont{author in head/foot}%
    \hfill\insertshortauthor
  \end{beamercolorbox}%
  \begin{beamercolorbox}%
    [wd=.5\paperwidth,ht=2.5ex,dp=1.125ex,leftskip=.3cm ,rightskip=.3cm]%
    {title in head/foot}%
    \usebeamerfont{title in head/foot}%
    \insertshorttitle\hfill\insertframenumber{}
  \end{beamercolorbox}%
}%

\setbeamertemplate{caption}{\raggedright\insertcaption\par}
\captionsetup[figure]{font=small,skip=0pt}
\begin{frame}
\titlepage
\end{frame} 

\begin{frame}
\frametitle{Gliederung}
\tableofcontents
\end{frame} 

\newcommand{\tmotiv}{Der Zerfall $D\rightarrow K l^+ \nu_l$}
\section{\tmotiv}
% 
 \begin{frame}
   \frametitle{Standardmodell}
   \pause
   \begin{minipage}[h]{0.48\textwidth}
    \begin{itemize}
    \visible<2->{\item Teilcheninhalt:}
    \begin{itemize}
     \visible<3->{\item Leptonen: $e^+$, $\mu^+$, $\nu$}
     \visible<4->{\item Quarks: $\bar u$, $\bar d$, $s$, $c$}
     \visible<5->{\item Vektorbosonen: $W^+$, $g$}
     \visible<6->{\item (Skalarbosonen: $H$, $\pi$)}
    \end{itemize}  
    \visible<2->{\item Fundamentale Wechselwirkungen:} 
    \begin{itemize}
     \visible<7->{\item starke Wechselwirkung (QCD)}
     \visible<8->{\item elektroschwache Wechselwirkung (GSW-Theorie)}
    \end{itemize}  
   \end{itemize}
   \end{minipage}
   \begin{minipage}[h]{0.48\textwidth}
   \begin{figure}[h]
    \includegraphics[width = 2\textwidth]{../Abbildungen/Teilchen.png}
   \end{figure}
   \end{minipage}
 \end{frame}
 
 \begin{frame}
  \frametitle{Feynmangraph}
   \setcounter{framenumber}{4}
 \begin{minipage}[h]{0.38\textwidth}
  \begin{enumerate}
   \item ruhendes $D$-Meson
 \end{enumerate}
 \end{minipage}
 \begin{minipage}[h]{0.58\textwidth}
   \begin{figure}[h]
   \includegraphics[width = 1.9\textwidth]{../Abbildungen/DFeyn1.png}
    \caption{Feynmangraph des $D\rightarrow Kl\nu$ Zerfalls}
  \end{figure}
 \end{minipage}
 \end{frame}
 
 \begin{frame}
  \frametitle{Feynmangraph}
 \setcounter{framenumber}{4}
 \begin{minipage}[h]{0.38\textwidth}
  \begin{enumerate}
   \item ruhendes $D$-Meson
   \item propagiert in $t$
  \end{enumerate}
 \end{minipage}
 \begin{minipage}[h]{0.58\textwidth}
   \begin{figure}[h]
   \includegraphics[width = 1.9\textwidth]{../Abbildungen/DFeyn2.png}
    \caption{Feynmangraph des $D\rightarrow Kl\nu$ Zerfalls}
  \end{figure}
 \end{minipage}
 \end{frame}
 
 \begin{frame}
  \frametitle{Feynmangraph}
  \setcounter{framenumber}{4}
 \begin{minipage}[h]{0.38\textwidth}
  \begin{enumerate}
   \item ruhendes $D$-Meson
   \item propagiert in $t$.
   \item $c$ wandelt unter Abstrahlung von $W^+$ in $s$ 
  \end{enumerate}
 \end{minipage}
 \begin{minipage}[h]{0.58\textwidth}
   \begin{figure}[h]
   \includegraphics[width = 1.9\textwidth]{../Abbildungen/DFeyn3.png}
    \caption{Feynmangraph des $D\rightarrow Kl\nu$ Zerfalls}
  \end{figure}
 \end{minipage}
 \end{frame}
 
 \begin{frame}
  \frametitle{Feynmangraph}
  \setcounter{framenumber}{4}
 \begin{minipage}[h]{0.38\textwidth}
  \begin{enumerate}
   \item ruhendes $D$-Meson
   \item propagiert in $t$.
   \item $c$ wandelt unter Abstrahlung von $W^+$ in $s$
   \item $W^+$ zerstrahlt in Leptonpaar $l^+$, $\nu_l$
  \end{enumerate}
 \end{minipage}
 \begin{minipage}[h]{0.58\textwidth}
   \begin{figure}[h]
   \includegraphics[width = 1.9\textwidth]{../Abbildungen/DFeyn4.png}
    \caption{Feynmangraph des $D\rightarrow Kl\nu$ Zerfalls}
  \end{figure}
 \end{minipage}
 \end{frame}
 
 \begin{frame}
 \frametitle{\"Uberblick}
 \pause
  \begin{minipage}[h]{0.62\textwidth}
  \begin{itemize}
   \visible<2->{\item Fermis Goldene Regel für Zerfälle:}\visible<3->{ $\underbrace{\mathrm{d}\Gamma}_{\text{Breite}} = \frac{1}{2m_D} \underbrace{\mathrm{d}\Phi}_{\text{Phasenraum}}\, \cdot \,{\underbrace{|M|}_{\text{Amplitude}}}^2$}
   \visible<2->{\item Teilchenstr\"ome}
   \begin{itemize}
    \visible<4->{\item [$\circ$] relativistischer Dirac-Strom}
    \visible<5->{\item [$\circ$] kurze Reichweite von $W^+$ f\"ur geringe Energien \\$\rightarrow$ Beschreibung durch 4-Fermionen-Wechselwirkung}
   \end{itemize}
 
   \visible<2->{\item Starke WW zwischen $c$ und $\bar q_1$}
   \begin{itemize}
    \visible<6->{\item[$\circ$] erh\"alt Parit\"at $\mathcal{P}$}
    \visible<7->{\item[$\circ$] st\"orungsrechnerisch nicht erfassbar \\$\rightarrow$ Darstellung durch \textbf{Formfaktoren} $f$}
   \end{itemize}
  \end{itemize}
 \end{minipage}
 \visible<1->{\begin{minipage}[h]{0.34\textwidth}
   \begin{figure}[h]
   \includegraphics[width = 1.3\textwidth]{../Abbildungen/DFeynSpez.png}
%     \caption{Feynmangraph des $D\rightarrow Kl\nu$ Zerfalls}
  \end{figure}
 \end{minipage}}
 \end{frame}


\section{Fermis Goldene Regel}
\begin{frame}
\tableofcontents[currentsection]
\end{frame}
\subsection{Differentielle Zerfallsbreite d$\Gamma$}
 \begin{frame}
    \frametitle{Zerfallsbreite}
    \pause
    \begin{itemize}
     \visible<2->{\item Inverses der hier sehr kurzen Lebensdauer $\tau$}
     \visible<3->{\item Energiemessung führt wegen Energieunschärfe zu Verteilungen}
     \visible<4->{\item [$\rightarrow$] Breite der Verteilung $\Gamma$ kann gemessen werden}
     \end{itemize}
 \end{frame}

\begin{frame}
 \frametitle{Ergebnis der differentiellen Zerfallsbreite}
 Fermis Goldene Regel:\\
 \pause
 \begin{align*}
 \visible<2->{\mathrm{d} \Gamma(D\rightarrow Kl\nu) &= \frac{|M|^2}{2m_D} \mathrm{d}\Phi(K,\,l,\,\nu)} \nonumber\\
 \visible<3->{&= \frac{G_F^2 |V_{cs}|^2}{24\pi^3}|f_+(q^2)|^2|p_K|^3\mathrm{d}q^2}
 \label{eq_theoGamma}
 \end{align*}
 \vskip-1.5em
 \vspace{0.5cm}
 \visible<3->
 {\small{Fermikonstante $G_F$,\\ CKM-Element $V_{cs}$,\\ Formfaktor $f_+$,\\Kaonimpuls $p_K$,\\ Impuls\"ubertrag $q^2$}}
\end{frame}



\subsection{Phasenraumvolumen d$\Phi$}
\begin{frame}
\frametitle{Phasenraumvolumen}
 \begin{itemize}
  \visible<2->{\item Enthält kinematische Informationen (Energien, Impulse)}
  \visible<3->{\item Je mehr Endzustände existieren, umso größer ist $\Phi$}
  \visible<4->{\item Nicht vom Matrixelement unabhängig berechenbar, da es Viererimpulse enthält}
 \end{itemize}
 \visible<5->{Ein erster Ausdruck:
 \begin{align*}
  \mathrm{d} \Phi & = (2\pi)^4\frac{\mathrm{d}^3p_K}{2(2\pi)^3E_K} \frac{\mathrm{d}^3k_1}{2(2\pi)^3E_1} \frac{\mathrm{d}^3k_2}{2(2\pi)^3E_2}\delta^4(p_D-p_K-k_1-k_2)\nonumber
 \end{align*}}
\end{frame}

\subsection{Matrixelement $M$}
\begin{frame}
\frametitle{Matrixelement}
 \pause
 \begin{itemize}
  \visible<2->{\item Enthält dynamische Informationen (Wechselwirkungen)}
  \visible<3->{\item Beschreibt Übergang ähnlich Streuung von Startzustand $i$ zu Endzustand $f$}
  \visible<4->{\item Betragsquadrat $|M|^2$ kann als Wahrscheinlichkeit für Reaktion betrachtet werden}
 \end{itemize}
  \visible<5->{Ein erster Ausdruck:
  \begin{align*}
   M = \langle Kl\nu\,|\mathcal{H}|\,D\rangle
  \end{align*}}
\end{frame}


\section{Teilchenstr\"ome}
\begin{frame}
\tableofcontents[currentsection]
\end{frame}

\subsection{Dirac-Gleichung}
\begin{frame}
\frametitle{Dirac-Gleichung}
\framesubtitle{Dirac-Gleichung}
\pause
\begin{itemize}
 \visible<2->{\item Lorentzinvariant}
 \visible<3->{\item Für Spin $\sfrac12$ -Teilchen}
 \visible<4->{\item Besitzt positiv definite Wahrscheinlichkeitsdichte $j^0$}
\end{itemize}
\visible<5-> {Dirac-Gleichung:
\begin{align*}
 (i\gamma_\mu\partial^\mu-m)\psi = (i\slashed{\partial}-m)\psi=(\slashed{p}-m)\psi = 0
 %\label{eq_dirac}
\end{align*}}
\vspace{0.5cm}

\visible<5->{\small{Dirac-Matrix $\gamma^\mu$,\\
Dirac-Wellenfunktion $\psi$,\\
Dirac-Spinoren $u$, $v$}}
\end{frame}

\begin{frame}
\frametitle{Dirac-Gleichung}
 \framesubtitle{Dirac-Strom $j^\mu$}
 \pause
 \begin{itemize}
  \visible<2->{\item Beschreibt Wahrscheinlichkeitsstrom eines propagierenden Teilchens}
  \visible<3->{\item Strom genügt Kontinuitätsgleichung $\partial_\mu j^\mu = 0$}
 \end{itemize}
\visible<4->{
 \begin{align*}
  j^\mu = \bar \psi \gamma^\mu \psi. 
 \end{align*}}
\end{frame}


\subsection{4-Fermionen-Wechselwirkung}


\begin{frame}
\frametitle{4-Fermionen-Wechselwirkung}
\framesubtitle{Rechtfertigung}
 \pause
  \begin{minipage}[h]{0.58\textwidth}
 \begin{itemize}
  \visible<2->{\item Am $W^+$ koppeln hier ein hadronischer  und ein leptonischer Strom}
  \visible<3->{\item Hohe Masse des $W^+$ ($\approx$ 82 GeV) heißt kurze Lebensdauer
  \begin{itemize}
  \item[$\rightarrow$] Vier Fermionen(-ströme) wechselwirken in einem Punkt
  \end{itemize}}
  \visible<5->{\item Niederenergetischer Grenzfall ($q^2<2\GeV^2$) der GSW-Theorie}
 \end{itemize}
  \end{minipage}
  \begin{minipage}[h]{0.38\textwidth}
  \visible<4->{\begin{figure}[h]
   \includegraphics[width = 3\textwidth]{../Abbildungen/4FermiPraes.png}
  \end{figure}}
  \end{minipage}
\end{frame}

 \begin{frame}
 \frametitle{4-Fermionen-Wechselwirkung}
  \framesubtitle{V-A-Theorie}
  \visible<2->{Aus Paritätsverletzung ($\rightarrow$ V-A-Theorie)
  folgt Erweiterung des Dirac-Stroms um Axialvektoranteil:}
  \visible<3->{\begin{align*}
   j^\mu = \bar \psi \gamma^\mu (1-\gamma_5) \psi
  \end{align*}}
% 
%   \begin{itemize}
%    \visible<2->{\item Ströme haben diverses Verhalten unter Lorentz-Transformationen (S, P, V, A, T)}
%    \visible<3->{\item Experimente erfordern \textbf{Paritätsverletzung} (Schwache WW koppelt an linkshändige Teilchen und rechtshändige Antiteilchen)}
%   \end{itemize}

 \end{frame}
 
%  \begin{frame}
%  \frametitle{Strom-Strom-Kopplung}
%  \begin{itemize}
%    \visible<1->{\item Dies erfordert pseudoskalaren, also kontrahierten $\mathcal{H}$ \\$\rightarrow$ Vektorstrom-Axialvektorstrom-Kopplung (V-A) }
%    \visible<2->{\item Projektionsoperator $P=(1-\gamma_5)$ extrahiert linkshändige Komponente der Spinoren}
%    \begin{itemize}
%     \visible<3->{\item [$\rightarrow$] Dirac-Strom wird um Axialvektorstromanteil erweitert:}
%    \end{itemize}
%  \end{itemize}
%    \visible<3->{\begin{align*}
%    j^\mu = \bar \psi \gamma^\mu (1-\gamma_5) \psi
%   \end{align*}}
%  \end{frame}

 
% \begin{frame}
%  \frametitle{C-Matrix $V_\text{C}$}
%  \begin{itemize}
%   \item Schwache Wechselwirkung ändert Flavourquantenzahlen
%   \item Schwierigkeiten, $s$ und $c$ zu Dublett zu formen, werden durch Cabibbo-Drehung erklärt
%   \begin{itemize}
%    \item [$\rightarrow$] Darstellung des hadronischen Stroms durch Drehmatrix $V_\text{C}$
%   \end{itemize}
%  \end{itemize}
%  \begin{align*}
%   j^\mu = (\bar u \bar c)\gamma^\mu(1-\gamma_5)\underbrace{\begin{pmatrix}
% 						  \,\,\,\,\cos \theta_c\quad \sin \theta_c\\
% 						  -\sin \theta_c\quad \cos \theta_c
% 					       \end{pmatrix}}_{V_\text{C}} \begin{pmatrix}
% 							      d\\
% 							      s
% 							      \end{pmatrix}
% \end{align*}
% 
% \vspace{1cm}
% Cabibbo-Winkel $\theta_c \approx 13^\circ$
% \end{frame}

\begin{frame}
\frametitle{4-Fermionen-Wechselwirkung}
 \framesubtitle{CKM-Matrix $V_\text{CKM}$}
 \pause
 \begin{itemize}
  \visible<2->{\item Schwache WW ändert Flavourquantenzahlen und \textbf{verletzt CP}}
  \visible<3->{\item Ausdruck für Übergangswahrscheinlichkeit von Quarks in Form einer (vermutlich) unitären 3$\times$3 Matrix}
 \end{itemize}
\visible<4->{\begin{align*}
 V_{\text{CKM}} = \begin{pmatrix}
			    V_{ud} & V_{us}& V_{ub}\\
			    V_{cd} & \color{red}{V_{cs}} & V_{cb}\\
			    V_{td} & V_{ts} & V_{tb}
			    \end{pmatrix}
\end{align*}}

\end{frame}

\begin{frame}
\frametitle{4-Fermionen-Wechselwirkung}
 \framesubtitle{CKM-Matrix $V_\text{CKM}$}
  \visible<1->{Ausgedrückt in der Wolfensteinparametrisierung:
 \begin{align*}
 V_{\text{CKM}} = \begin{pmatrix}
			    1-\sfrac12\,\lambda^2 & \lambda & \lambda^3A(\rho-\text{i}\eta)\\
			    -\lambda & \color{red}{1-\sfrac12\, \lambda^2} &\lambda^2A\\
			    \lambda^3A(1-\rho-\text{i}\eta) &-\lambda^2A & 1
			    \end{pmatrix} \, +  \, \mathcal{O}(\lambda^4)
\end{align*}}
 \begin{itemize}
 \visible<2->{\item $V_{\text{CKM}}$ enthält nun komplexe Phase zur Erklärung der CP-Verletzung}
 \visible<2->{\item und drei Eulerwinkel $\theta_{12} = \theta_c,\, \theta_{13}$ und $\theta_{23}$}
\end{itemize}
\vspace{0.4cm}
\visible<2->{\small{Cabibbo-Winkel $\theta_c \approx 13^\circ$, \\$\lambda = \sin\theta_{12} \approx 0,2$ ,\\ $A\lambda^2 = \sin\theta_{23}$, \\ $A\lambda^3(\rho-\text{i}\eta) = \sin\theta_{13}\mathbf{e}^{-\text{i}\phi}$}}

\end{frame}



\section{Formfaktoren}
\begin{frame}
\tableofcontents[currentsection]
\end{frame}
\begin{frame}
\frametitle{Formfaktoren}
 \framesubtitle{Motivation}
 \pause
 \begin{itemize}
  \visible<2->{\item Fermi-Wechselwirkung berücksichtigt die starke WW zwischen $c$ und $\bar q_1$ nicht
  \begin{itemize}
   \visible<3->{\item [$\rightarrow$] Leptonenstrom weiterhin dadurch beschrieben}
   \visible<3->{\item [$\rightarrow$] Hadronenstrom durch Formfaktoren darstellen}
  \end{itemize}}
  \visible<4->{\item Formfaktoren sind einheitenlose Größen, die theoretisch unzugängliche Einflüsse enthalten (sollen berechnet werden)}
  \visible<5->{\item Extraktion des CKM-Elements oder auch Bestimmung des kontinuierlichen Verlaufs möglich}
 \end{itemize}
\end{frame}

\begin{frame}
\frametitle{Formfaktoren}
 \framesubtitle{Motivation}
 \begin{itemize}
  \visible<1->{\item Viererimpulse $p_D$ und $p_K$ sind einzige Freiheitsgrade und müssen zur Darstellung ausreichen}
  \visible<2->{\item Da QCD Parität erhält, müssen Formfaktorausdrücke dasselbe Transformationsverhalten unter Parität haben, wie $V$ bzw. $A$.}
 \end{itemize}

\end{frame}

\begin{frame}
\frametitle{Formfaktoren}
 \framesubtitle{Kinematische Grenzen}
 \pause
 \visible<2->{Aus der Viererimpulserhaltung ergeben sich die Grenzen für $q^2$, die den Bereich für den Fit von $f_+$ angeben:}
 \begin{align*}
 \visible<3->{p_D^\mu &= p_K^\mu + k_l^\mu + k_\nu^\mu \\
 p_D^\mu - p_K^\mu &=: q^\mu := k_l^\mu + k_\nu^\mu }\\
 \visible<4->{\left(p_D^\mu-p_K^\mu\right)^2 &= q^2 =  (k_l^\mu + k_\nu^\mu )^2\\
 m_D^2 + m_K^2 - 2m_DE_K &= q^2 = m_l^2 + m_\nu^2 + E_lE_\nu - |\vec k_l||\vec k_\nu|\cos(\xi)}
\end{align*}
\visible<5->{Hieraus ergeben sich bei abermals vernachlässigbaren Leptonenmassen ($E = |\vec k|$) die Bereichsgrenzen zu
\begin{align*}
 0 \leq q^2 \leq (m_D-m_K)^2.
\end{align*}}

\vspace{-0.1cm}
\visible<4->{Leptonenzwischenwinkel $\xi$}
\end{frame}


\subsection{Axialvektorformfaktoren und $f_-$}
\begin{frame}
\frametitle{Axialvektorformfaktoren und $f_-$}
\framesubtitle{Axialvektorformfaktoren}
\pause
\begin{itemize}
 \visible<2->{\item Eigenwerte der Parität $\mathcal{P}$ sind $\pi = \pm1$ und multiplikativ, da diskrete Symmetrie}
 \visible<3->{\item Vektoren und Pseudoskalare transformieren mit $\pi = -1$, Axialvektoren mit $\pi = +1$}
\end{itemize}
\visible<4->{\begin{align}
 \mathcal{P} \, \big\langle\bar K^0\,\big|V^\mu|\,D^+\big\rangle &= (-1)\cdot(-1)\cdot(-1) = -1 \nonumber \\
 \mathcal{P} \, \big\langle\bar K^0\,\big|A^\mu|\,D^+\big\rangle &= (-1)\cdot(+1)\cdot(-1) = +1 \nonumber
\end{align}}

\end{frame}

\begin{frame}
\frametitle{Axialvektorformfaktoren und $f_-$}
 \framesubtitle{Axialvektorformfaktoren}
 \visible<1->{\begin{align}
 \mathcal{P} \, \big\langle\bar K^0\,\big|V^\mu|\,D^+\big\rangle &= (-1)\cdot(-1)\cdot(-1) = -1 \nonumber \\
 \mathcal{P} \, \big\langle\bar K^0\,\big|A^\mu|\,D^+\big\rangle &= (-1)\cdot(+1)\cdot(-1) = +1 \nonumber
\end{align}}
 \begin{itemize}
 \visible<1->{\item Keine Kombination aus $p_D^\mu$, $p_K^\mu$ und dem Levi-Civita-Tensor $\epsilon^{\mu\nu\alpha\beta}$ transformiert mit $\pi = +1$}
 \visible<2->{\begin{itemize}
  \item [$\rightarrow$] $\big\langle K(p_K)\,\big|A^\mu\big|\, D(p_D)\big\rangle = 0$
  \item [$\rightarrow$] Keine Axialvektorformfaktoren!
 \end{itemize}}
\end{itemize}
\end{frame}


\begin{frame}
\frametitle{Axialvektorformfaktoren und $f_-$}
 \framesubtitle{Vektorformfaktoren}
 \pause
 \visible<2->{Viererimpulse selbst transformieren unter Parität wie Vektoren}
   \visible<3->{\begin{itemize}
   \item [$\rightarrow$] Allgemeine Darstellung durch zwei Formfaktoren $f_+$, $f_-$:
  \end{itemize}}

 \visible<4->{\begin{align*}
 \big\langle K(p_K)\,\big|V^\mu\big|\, D(p_D)\big\rangle = f_+(q^2)(p_D+p_K)^\mu + f_-(q^2)(p_D-p_K)^\mu
\end{align*}}
\end{frame}

\begin{frame}
\frametitle{Axialvektorformfaktoren und $f_-$}
 \framesubtitle{Formfaktor $f_-$}
 \pause
 \visible<2->{Betrachtung von $M_-$ nur mit $f_-$:}
 \begin{align*}
  \visible<2->{M_- &= \frac{G_F V_{cs}}{\sqrt{2}} f_-(q^2)(p_D-p_K)^\mu\bar u_\nu \gamma_\mu(1-\gamma_5)v_l}\\
 \visible<3->{&= \frac{G_F V_{cs}}{\sqrt{2}} f_-(q^2)(k_\nu+k_l)^\mu \bar u_\nu \gamma_\mu(1-\gamma_5)v_l}\\
  \visible<4->{&=\frac{G_F V_{cs}}{\sqrt{2}} f_-(q^2)\bar u_\nu (\slashed{k}_\nu + \slashed{k}_l) (1-\gamma_5)v_l}\\
 \visible<5->{&\stackrel{\text{Dirac}}{=}\frac{G_F V_{cs}}{\sqrt{2}} f_-(q^2)[m_\nu \bar u_\nu (1-\gamma_5)v_l + m_l \bar u_\nu (1+\gamma_5)v_l]} 
 \end{align*}
\visible<6->{Die Leptonmassen sind für $l=e,\,\mu$ verglichen mit $m_D$ vernachlässigbar
\begin{itemize}
 \item [$\rightarrow$] $f_-$ liefert ebenfalls keinen Beitrag!
\end{itemize}}

\end{frame}


\subsection{Formfaktor $f_+$}


\begin{frame}
\frametitle{Formfaktor $f_+$}
\framesubtitle{Parametrisierung}
\pause
\begin{itemize}
 \visible<2->{\item $\mathrm{d}\Gamma \propto |f_+(q^2)|^2\mathrm{d}q^2$}
 \visible<3->{\item Bei bestimmten Energien divergiert $\Gamma$
 \begin{itemize}
  \item [$\rightarrow$] Pol-Verhalten um $q^2 = m_{D^*}^2$ (außerhalb des phys. rel. Bereichs)
 \end{itemize}}
 \visible<4->{\item Reihenentwicklung um $q^2$=0 konvergiert nicht gut} 
 \visible<5->{\begin{itemize}
  \item [$\rightarrow$] Variablentransformation von $q^2$ nach $z$ in die komplexe Zahlenebene, wobei $|z|\stackrel{!}{<}1$
 \end{itemize}}
 \visible<6->{\item [$\rightarrow$] Parametrisierung durch Pol und Polynomreihe in $z$}
\end{itemize}
\end{frame}

\begin{frame}
\frametitle{Formfaktor $f_+$}
 \framesubtitle{Parametrisierung}
\pause
 \begin{minipage}[h]{0.48\textwidth}
 \visible<2->{Eine Parametrisierung für $f_+$ mit diesen Eigenschaften lautet:
\begin{align*}
 &f_+(q^2) = \frac{1}{1-\frac{q^2}{m_{D^*}^2}} \sum\limits_{i=0}^\infty a_i\,z^i(t_0,\, q^2)\\
 &z(t_0,\, q^2)= \frac{\sqrt{t_+-q^2}-\sqrt{t_+-t_0}}{\sqrt{t_+-q^2}+\sqrt{t_+-t_0}}
\end{align*}}

\vspace{0.5cm}
\visible<2->{\small{$t_\pm$ = ($m_D \pm m_K$)$^2$,\\ $t_0$: $0\leq t_0 < t_+$}}

 \end{minipage}
 \begin{minipage}[h]{0.40\textwidth}
 \visible<2->{\begin{figure}
  \includegraphics[width=2.7\textwidth]{../Abbildungen/formMuster.png}
 \end{figure}}
 \end{minipage}
\end{frame}

\begin{frame}
\frametitle{Formfaktor $f_+$}
 \framesubtitle{Parametrisierung}
 \begin{itemize}
 \visible<1->{\item Polynomordnung $\mathcal{O}$ liefert Anzahl der Fitparameter $a_i$ }
 \visible<2->{\item Freier Parameter $t_0$ minimiert Fehler der $a_i$ \\($t_\text{opt} := t_+(1-\sqrt{1-t_-/t_+})$ minimiert Maximalwert von $|z|$)}
 \visible<3->{\item [$\rightarrow$] Variation in diesen beiden möglich}
 \end{itemize}
\end{frame}



\begin{frame}
\frametitle{Formfaktor $f_+$}
 \framesubtitle{Methode der kleinsten Quadrate}
 \pause
 \begin{itemize}
  \visible<2->{\item Fitfunktionen weisen Abweichungen von Messwerten auf}
  \visible<3->{\item Die quadrierten Abweichungen werden aufsummiert als $\chi^2$ bezeichnet}
  \visible<4->{\begin{itemize}
   \item [$\rightarrow$] Fitparameter werden variiert, bis $\chi^2$ minimal ist
  \end{itemize}}
 \end{itemize}
 \visible<5->{Die hier verwandte $\chi^2$-Funktion lautet
 \begin{align*}
  \chi^2 = \sum\limits_{i,j=1}^m (\Delta \Gamma_i - g_i(f_+))C^{-1}_{ij}(\Delta \Gamma_j - g_j(f_+))
 \end{align*} 
 und wird durch ein \texttt{Python}-Skript unter Verwendung des Minimierungsmoduls \texttt{Minuit} vom CERN minimiert.}
\end{frame}

\begin{frame}
\frametitle{Formfaktor $f_+$}
 \framesubtitle{Methode der kleinsten Quadrate}
  \visible<1->{\begin{align*}
  \chi^2 = \sum\limits_{i,j=1}^m (\Delta \Gamma_i - g_i(f_+))C^{-1}_{ij}(\Delta \Gamma_j - g_j(f_+))
 \end{align*}}
 \begin{itemize}
   \visible<2->{\item Anzahl diskreter Intervalle $m$ ($q^2$-Bins)}
   \visible<3->{\item Kovarianzmatrix $C = C^{\text{stat}} + C^{\text{sys}}$; $C^{\alpha}_{ij} = \sigma^{\alpha}_i \sigma^{\alpha}_j \cdot \rho^{\alpha}_{ij}$\\ 
  $\alpha = \text{stat, sys}$; Varianzen $\sigma$; Korrelationsmatrix $\rho$  }
 \end{itemize}
\end{frame}

\begin{frame}
\frametitle{Formfaktor $f_+$}
 \framesubtitle{Methode der kleinsten Quadrate}
  \visible<1->{\begin{align*}
  \chi^2 = \sum\limits_{i,j=1}^m (\Delta \Gamma_i - g_i(f_+))C^{-1}_{ij}(\Delta \Gamma_j - g_j(f_+))
 \end{align*}}
 \begin{itemize}
   \visible<2->{\item experimentell erfasste Daten $\Delta \Gamma$ (CLEO Collaboration)}
   \visible<3->{\item theoretische Werte $g = \frac{G_F^2 |V_{cs}|^2}{24\pi^3}\int|p_K(q_i^2)|^3 \cdot |\color{blue}f_+(q^2_i)\color{black}|^2 \mathrm{d} q_i^2$}
 \end{itemize}
\end{frame}





\section{Resultate für $f_+$}
\begin{frame}
\tableofcontents[currentsection]
\end{frame}

\begin{frame}
 \frametitle{Vorbereitung}
 \pause
 \begin{itemize}
  \visible<2->{\item Ergebnisse für $f_+$ mit Werten der CLEO Collaboration \\(Lepton = Positron)}
  \visible<3->{\item Für $D^+\rightarrow \bar K^0 e^+ \nu_e$ Betrachtung der Variation von $t_0$ und $\mathcal{O}$}
  \visible<4->{\item Bei $D^0\rightarrow K^- e^+ \nu_e$ gelten entsprechende Einflüsse analog, nur nicht so deutlich}
 \end{itemize}
\end{frame}

\begin{frame}
 \frametitle{Repräsentativer Fit für $D^0\rightarrow K^- e^+ \nu_e$}
  \begin{minipage}[h]{0.66\textwidth}
  \includegraphics[width=1.0\textwidth]{../Fit/D0-2O-topt.pdf}
 \end{minipage}
 \begin{minipage}[h]{0.32\textwidth}
  \begin{table}[h]
   \begin{tabular}{c|l}
   \toprule
     & Wert\\
    \midrule
    $t_0$ & $t_\text{opt}$\\
    $\mathcal{O}$ & 2\\
    \midrule
    $a_0$ & 0,744(7)\\
    $a_1$ & -0,775(257)\\
    $a_2$ & 7,876(6,691)\\
    \midrule
    $\chi^2$ & 2,9\\
    $f_+(0)|V_{cs}|$ & 0,725\\
    \bottomrule\bottomrule
   \end{tabular}

  \end{table}

 \end{minipage}
\end{frame}


\begin{frame}
 \frametitle{Variation in $t_0$ für $D^+\rightarrow \bar K^0 e^+ \nu_e$}
 \begin{minipage}[h]{0.66\textwidth}
  \includegraphics[width=1.0\textwidth]{../Fit/D+-2O-tmin.pdf}
 \end{minipage}
 \begin{minipage}[h]{0.32\textwidth}
  \begin{table}[h]
   \begin{tabular}{c|l}
   \toprule
     & Wert\\
    \midrule
    $t_0$ & $t_-$\\
    $\mathcal{O}$ & 2\\
    \midrule
    $a_0$ & 0,714(37)\\
    $a_1$ & 1,22(1,17)\\
    $a_2$ & -12,67(8,90)\\
    \midrule
    $\chi^2$ & 12,0\\
    $f_+(0)|V_{cs}|$ & 0,707\\
    \bottomrule\bottomrule
   \end{tabular}

  \end{table}

 \end{minipage}
\end{frame}


\begin{frame}
 \frametitle{Variation in $t_0$ für $D^+\rightarrow \bar K^0 e^+ \nu_e$}
 \begin{minipage}[h]{0.66\textwidth}
  \includegraphics[width=1.0\textwidth]{../Fit/D+-2O-0.pdf}
 \end{minipage}
 \begin{minipage}[h]{0.32\textwidth}
  \begin{table}[h]
   \begin{tabular}{c|l}
   \toprule
     & Wert\\
    \midrule
    $t_0$ & 0\\
    $\mathcal{O}$ & 2\\
    \midrule
    $a_0$ & 0,707(14)\\
    $a_1$ & -1,356(685)\\
    $a_2$ & -12,58(8,81)\\
    \midrule
    $\chi^2$ & 12,0\\
    $f_+(0)|V_{cs}|$ & 0,707\\
    \bottomrule\bottomrule
   \end{tabular}

  \end{table}

 \end{minipage}
\end{frame}

\begin{frame}
 \frametitle{Variation in $t_0$ für $D^+\rightarrow \bar K^0 e^+ \nu_e$}
 \begin{minipage}[h]{0.66\textwidth}
  \includegraphics[width=1.0\textwidth]{../Fit/D+-2O-topt.pdf}
 \end{minipage}
 \begin{minipage}[h]{0.32\textwidth}
  \begin{table}[h]
   \begin{tabular}{c|l}
   \toprule
     & Wert\\
    \midrule
    $t_0$ & $t_\text{opt}$\\
    $\mathcal{O}$ & 2\\
    \midrule
    $a_0$ & 0,744(11)\\
    $a_1$ & -0,071(324)\\
    $a_2$ & -12,56(8,78)\\
    \midrule
    $\chi^2$ & 12,0\\
    $f_+(0)|V_{cs}|$ & 0,707\\
    \bottomrule\bottomrule
   \end{tabular}

  \end{table}

 \end{minipage}
\end{frame}


\begin{frame}
 \frametitle{Variation in $\mathcal{O}$ für $D^+\rightarrow \bar K^0 e^+ \nu_e$}
  \begin{minipage}[h]{0.66\textwidth}
  \includegraphics[width=1.0\textwidth]{../Fit/D+-2O-topt.pdf}
 \end{minipage}
 \begin{minipage}[h]{0.32\textwidth}
  \begin{table}[h]
   \begin{tabular}{c|l}
   \toprule
     & Wert\\
    \midrule
    $t_0$ & $t_\text{opt}$\\
    $\mathcal{O}$ & 2\\
    \midrule
    $a_0$ & 0,744(11)\\
    $a_1$ & -0,071(324)\\
    $a_2$ & -12,56(8,78)\\
    \midrule
    $\chi^2$ & 12,0\\
    $f_+(0)|V_{cs}|$ & 0,707\\
    \bottomrule\bottomrule
   \end{tabular}

  \end{table}

 \end{minipage}
\end{frame}

\begin{frame}
 \frametitle{Variation in $\mathcal{O}$ für $D^+\rightarrow \bar K^0 e^+ \nu_e$}
  \begin{minipage}[h]{0.66\textwidth}
  \includegraphics[width=1.0\textwidth]{../Fit/D+-1O-topt.pdf}
 \end{minipage}
 \begin{minipage}[h]{0.32\textwidth}
  \begin{table}[h]
   \begin{tabular}{c|l}
   \toprule
     & Wert\\
    \midrule
    $t_0$ & $t_\text{opt}$\\
    $\mathcal{O}$ & 1\\
    \midrule
    $a_0$ & 0,739(10)\\
    $a_1$ & -0,421(207)\\
    $a_2$ & \\
    \midrule
    $\chi^2$ & 14,1\\
    $f_+(0)|V_{cs}|$ & 0,718\\
    \bottomrule\bottomrule
   \end{tabular}

  \end{table}

 \end{minipage}
\end{frame}

\begin{frame}
 \frametitle{Variation in $\mathcal{O}$ für $D^+\rightarrow \bar K^0 e^+ \nu_e$}
  \begin{minipage}[h]{0.66\textwidth}
  \includegraphics[width=1.0\textwidth]{../Fit/D+-0O-topt.pdf}
 \end{minipage}
 \begin{minipage}[h]{0.32\textwidth}
  \begin{table}[h]
   \begin{tabular}{c|l}
   \toprule
     & Wert\\
    \midrule
    $t_0$ & $t_\text{opt}$\\
    $\mathcal{O}$ & 0\\
    \midrule
    $a_0$ & 0,731(9)\\
    $a_1$ & \\
    $a_2$ & \\
    \midrule
    $\chi^2$ & 18,2\\
    $f_+(0)|V_{cs}|$ & 0,731\\
    \bottomrule\bottomrule
   \end{tabular}

  \end{table}

 \end{minipage}
\end{frame}

\begin{frame}
 \frametitle{Diskussion}
 \pause
 \visible<2->{Variation in $t_0$:}
 \begin{itemize}
  \visible<3->{\item Bei $t_\text{opt}$ ist im Verlauf von $q^2$ ein deutlich geringerer Fehlerschlauch erkennbar}
  \visible<4->{\item Für $f(0)|V_{cs}|$ bietet $t_0 = 0$ die geringste Varianz}
 \end{itemize}
 \visible<2->{Variation in $\mathcal{O}$:}
 \begin{itemize}
  \visible<5->{\item Höhere Ordnung bedingt größer werdende Fehlerschläuche wegen zunehmendem Beitrag weiterer Koeffizienten}
  \visible<6->{\item Ordnung manipuliert $f(0)|V_{cs}|$ nur geringfügig - erst in zweiter Nachkommastelle feststellbar}
 \end{itemize}
\end{frame}

\begin{frame}
 \frametitle{Diskussion}
 Parametrisierung und Formfaktor:
 \begin{itemize}
  \visible<2->{\item Graphen zeugen von einer die Messwerte gut beschreibenden Parametrisierung}
  \visible<3->{\item Wert für $f(0)|V_{cs}|$ ähnlich anderen Parametrisierungen, die bei CLEO aufgeführt sind}
  \visible<4->{\item Durch Rechnung Gitterquantenchromodynamik ergibt sich $f(0)$=0,73}
  \visible<5->{\item[$\rightarrow$] für $|V_{cs}|$ ergeben sich $|V_{cs,D^+}|=0,97$ und $|V_{cs,D^0}|=0,99$ \\ (vgl. Wolfenstein: $V_{cs} = 1-\sfrac12 \lambda^2 = 0,98$)}
 \end{itemize}

\end{frame}



\section{Ausblick}
\begin{frame}
\tableofcontents[currentsection]
\end{frame}

\begin{frame}
 \frametitle{Ausblick}
 \pause
 \visible<2->{Unitarität der CKM-Matrix}
 \begin{itemize}
  \visible<3->{\item Unitaritätsdreieck in komplexer Ebene erstellbar aus CKM-Elementen (Fläche ist Maß für CP-Verletzung)}
  \begin{itemize}
   \visible<5->{\item [$\rightarrow$] ist sie ausreichend für das Materie-Antimaterie-Ungleichgewicht?}
  \end{itemize}
  \visible<4->{\item CKM-Matrix 3-dimensional zur Erklärung der CP-Verletzung}
  \begin{itemize}
   \visible<6->{\item [$\rightarrow$]Sind alle Quarkflavour-Änderungsprozesse mit drei Generationen...}
   \visible<7->{\item [$\rightarrow$]... bzw. allein durch elektroschwache WW auch quantitativ beschreibbar?}
  \end{itemize} 
  \visible<8->{\item [$\rightarrow$] Verbesserte Ausmessung aller CKM-Elemente!}
 \end{itemize}

\end{frame}



\begin{frame}
\frametitle{Bonus}
\textbf{Bonusfolien}
\end{frame}

 \begin{frame}
  \frametitle{Berechnung der differentiellen Zerfallsbreite}
  \begin{align*}
   \mathrm{d}\Gamma = \frac{|M|^2}{2m_D} \mathrm{d}\Phi
  \end{align*}
 \end{frame}
 
\begin{frame}
  \frametitle{Berechnung der differentiellen Zerfallsbreite}
  \framesubtitle{Matrixelement}
  \begin{align*}
   M &= \langle Kl\nu\,|\mathcal{H}|\,D\rangle\\
   &= \frac{G_F V_{cs}}{\sqrt{2}} [f_+(q^2)P^\mu]\bar u(k_l) \gamma_\mu(1-\gamma_5)v(k_\nu) \\
\vspace{0.7cm}
   |M|^2 &= \frac{G_F^2 |V_{cs}|^2}{2} |f_+(q^2)^2| P^\mu P^\nu \underbrace{[\bar u(k_l)  \gamma_\mu(1-\gamma_5)v(k_\nu) ]^2}_{\text{Casimirs Trick}}\\
   & = \frac{G_F^2 |V_{cs}|^2}{2} |f_+(q^2)^2| P^\mu P^\nu \cdot 8\big(k_{l,\mu} k_{\nu,\nu} - g_{\mu\nu}k_lk_\nu + k_{l,\nu}k_{\nu,\mu}\big)\\
   & = 4G_F^2|V_{cs}|^2 |f_+(q^2)|^2 \color{green}{\big(2P^\mu P^\nu - P^2 g^{\mu\nu}\big)} \color{blue}{k_{l,\nu}k_{\nu,\mu}}
  \end{align*}

\end{frame}

\begin{frame}
   \frametitle{Berechnung der differentiellen Zerfallsbreite}
  \framesubtitle{Phasenraum}
   \begin{align*}
  \mathrm{d} \Phi & = \frac{1}{(2\pi)^5}\frac{\mathrm{d}^3p_K}{2E_K} \int \frac{\mathrm{d}^3k_1}{2E_1} \frac{\mathrm{d}^3k_2}{2E_2}\delta^4(p_D-p_K-k_1-k_2)\color{blue}{k_{l,\nu}k_{\nu,\mu}}
 \end{align*}
\end{frame}

\begin{frame}
    \frametitle{Berechnung der differentiellen Zerfallsbreite}
  \framesubtitle{Benutzte Gleichheiten}
\begin{align*}
 & \frac{\mathrm{d}^3p_K}{2E_K} = 2\pi |p_K|\mathrm{d}E_K\\
&|p_K| = \frac{\sqrt{\lambda(m_D^2, m_K^2, q^2)}}{2m_D}\\
&\int \frac{\mathrm{d}^3k_1}{2(2\pi)^3E_1} \frac{\mathrm{d}^3k_2}{2(2\pi)^3E_2}\delta^4(q-k_1-k_2)k_{1,\mu} k_{2,\nu} = \frac{\pi}{24}(q^2g_{\mu\nu} + 2q_\mu q_\nu)\\
\end{align*}
  
\end{frame}

\begin{frame}
     \frametitle{Berechnung der differentiellen Zerfallsbreite}
  \framesubtitle{Phasenraum}
  \begin{align*}
   \mathrm{d}\Phi = \frac{|p_K|\mathrm{d}E_K}{(2\pi)^4}\frac{\pi}{24}\color{red}{(q^2 g_{\mu\nu}+2q_\mu q_\nu)}
  \end{align*}

\end{frame}

\begin{frame}
     \frametitle{Berechnung der differentiellen Zerfallsbreite}
  \framesubtitle{Matrixelement und Phasenraum}
  Unter Verwendung von 
  \begin{align*}
  \color{green}{\big(2P^\mu P^\nu - P^2 g^{\mu\nu}\big)}\color{red}{(q^2 g_{\mu,\nu}+2q_\mu q_\nu)}\color{black} = 4 \lambda(m_D^2,m_K^2,q^2) = 16 m_D^2 |p_K|^2
  \end{align*}
  ergibt sich die oben aufgeführte Zerfallsbreite
  
\end{frame}





%   \begin{frame}
%   \begin{align}
%   \mathrm{d} \Phi & = (2\pi)^4\frac{\mathrm{d}^3p_K}{2(2\pi)^3E_K} \frac{\mathrm{d}^3k_1}{2(2\pi)^3E_1} \frac{\mathrm{d}^3k_2}{2(2\pi)^3E_2}\delta^4(p_D-p_K-k_1-k_2)\nonumber\\
%   &=\frac{1}{(2\pi)^5}\frac{\mathrm{d}^3p_K}{2E_K}\int \frac{\mathrm{d}^3k_1}{2(2\pi)^3E_1} \frac{\mathrm{d}^3k_2}{2(2\pi)^3E_2}\delta^4(q-k_1-k_2)k_{1,\mu} k_{2,\nu}\nonumber
%   \end{align}
%  Leptonimpulse $k_i$ integriert, da Verteilung der Zerfallsbreiten durch $\mathrm{d}q^2$ gemessen. Die Eintr\"age $k_{1,\mu}$ und $k_{2,\nu}$ stammen von $M$ (folgt im Anschluss)
%  \end{frame}
%  
%  \begin{frame}
%   \frametitle{Berechnung}
%   F\"ur die weitere Berechnung werden folgende Gleichungen des $D$-Meson-Ruhesystems ben\"otigt:
%   \begin{itemize}
%   \item $ \frac{\mathrm{d}^3p_K}{2E_K} = 2\pi |p_K|\mathrm{d}E_K$
%     \item $|p_K| = \frac{\sqrt{\lambda(m_D^2, m_K^2, q^2)}}{2m_D}$
%     \item $\int \frac{\mathrm{d}^3k_1}{2(2\pi)^3E_1} \frac{\mathrm{d}^3k_2}{2(2\pi)^3E_2}\delta^4(q-k_1-k_2)k_{1,\mu} k_{2,\nu} = \frac{\pi}{24}(q^2g_{\mu\nu} + 2q_\mu q_\nu)$
%   \end{itemize}
%   \vspace{0.7cm}
%   \small{$\lambda = a^2+b^2+c^2-2(ab+bc+ac)$ - K\"all\'en-Funktion,\\$g_{\mu,\nu} = \text{diag}(1,-1,-1,-1)$ - Minkowski-Metrik}
%  \end{frame}
% 
% \begin{frame}
%  \frametitle{Berechnung}
%  Diese f\"uhren zu einem Ausdruck f\"ur das Phasenraumvolumen:
%  \begin{align}
%   \mathrm{d}\Phi = \frac{\pi}{24}(q^2g_{\mu\nu} + 2q_\mu q_\nu)\frac{|p_K|}{(2\pi)^4}\mathrm{d}E_K
%  \end{align}
% \end{frame}
% 
% \begin{frame}
%  \frametitle{Berechnung II}
%  Abschliessend kann das quadrierte Matrixelement nach Umformung des leptonischen Anteils durch Casimirs Trick wie folgt geschrieben werden als
%  \begin{align}
%   |M|^2 & = \frac{G_F^2|V_{cs}|^2}{2}|f_+(q^2)|^2 P^\mu P^\nu \cdot 8\big(k_{l,\mu} k_{\nu,\nu} - g_{\mu\nu}k_lk_\nu + k_{l,\nu}k_{\nu,\mu}\big)\nonumber\\
%   & = 4G_F^2|V|^2 |f_+(q^2)|^2 \big(2P^\mu P^\nu - P^2 g^{\mu\nu}\big) k_{l,\nu}k_{\nu,\mu}
%   \label{eq_theoAmplitude}
%  \end{align}
% \end{frame}
% 
% \begin{frame}
% \frametitle{Berechnung II}
%  Nun k\"onnen \eqref{eq_theoPhase} und \eqref{eq_theoAmplitude} unter Verwendung von
%  \begin{align*}
%  \big(2P^\mu P^\nu - P^2 g^{\mu\nu}\big)\big(2q_\mu q_\nu + q^2g_{\mu\nu}) = 4 \lambda(m_D^2,m_K^2,q^2) = 16 m_D^2 |p_K|^2
%  \end{align*}
%  miteinander verkn\"upft und zu \eqref{eq_theoGamma} zusammengefasst werden
%  ++Verlinkung auf f++
% 
% \end{frame}




\end{document}
