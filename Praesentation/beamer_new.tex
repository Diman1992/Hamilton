

\documentclass[hyperref={pdfpagelabels=false}]{beamer}
% Die Hyperref Option hyperref={pdfpagelabels=false} verhindert die Warnung:
% Package hyperref Warning: Option `pdfpagelabels' is turned off
% (hyperref)                because \thepage is undefined. 
% Hyperref stopped early 
%
% \input{longheader.tex}
 %\input{global.tex}

\usepackage{lmodern}
% Das Paket lmodern erspart die folgenden Warnungen:
% LaTeX Font Warning: Font shape `OT1/cmss/m/n' in size <4> not available
% (Font)              size <5> substituted on input line 22.
% LaTeX Font Warning: Size substitutions with differences
% (Font)              up to 1.0pt have occurred.
%

% % % % % % % % % % % % % % % % % % % % % % % % % % % % % % % % % % % % % % % % % % % %
\usepackage{siunitx}
\sisetup{load-configurations=abbreviations}
\sisetup{
	%locale=DE,
	seperr=true,                    % Fehler anzeigen
	tightpm,                        % Abstand zwischen Fehler verringern
	tophrase={{\text{ bis }}},
	fraction=nice,
	per-mode=fraction,
	free-standing-units=true,
	space-before-unit=true,
	use-xspace=true,
	group-separator={{\text{~}}},
	list-final-separator={{\text{ und }}}
}
\usepackage{natbib}
\usepackage[labelformat=empty]{caption}
\usepackage{movie15}
\usepackage{xcolor,colortbl}
\usepackage{slashed}
\usepackage{amsfonts}
\usepackage{amssymb}
\usepackage{amsmath}
\usepackage{amscd}
\usepackage{amstext}
\usepackage[ngerman,german]{babel, varioref}
\usepackage[T1]{fontenc}
\usepackage[utf8]{inputenc}
\usepackage{xfrac}
\usepackage{booktabs}

\newcommand {\rot} {\; \mathrm{rot} \,}         % Rotation

\newcommand {\grad} {\; \mathrm{grad} \,}       % Gradient

\newcommand {\dive} {\; \mathrm{div} \,}        % Divergenz

\newcommand {\dx} {\; \mathrm{d} }              % Differential d

% % % % % % % % % % % % % % % % % % % % % % % % % % % % % % % % % % % % % % % % % % % % % % % % %
% Wenn \titel{\ldots} \author{\ldots} erst nach \begin{document} kommen,
% kommt folgende Warnung:
% Package hyperref Warning: Option `pdfauthor' has already been used,
% (hyperref) ... 
% Daher steht es hier vor \begin{document}

\title{Hamilton-Formalismus in der Beschleunigerphysik}  
\institute{
Technische Universit\"at Dortmund}
\author{Sonja Bartkowski, Dimitrios Skodras} 
\date{11.06.2015} 

% zusaetzlich ist das usepackage{beamerthemeshadow} eingebunden 
\usepackage{beamerthemeshadow}


%  \beamersetuncovermixins{\opaqueness<1>{25}}{\opaqueness<2->{15}}
%  sorgt dafuer das die Elemente die erst noch (zukuenftig) kommen 
%  nur schwach angedeutet erscheinen 
\beamersetuncovermixins{\opaqueness<1>{25}}{\opaqueness<2->{15}}
% klappt auch bei Tabellen, wenn teTeX verwendet wird\ldots

\beamertemplatenavigationsymbolsempty

\begin{document}

\setbeamertemplate{footline}
{%
  \leavevmode%
 \begin{beamercolorbox}%
    [wd=.5\paperwidth,ht=2.5ex,dp=1.125ex,leftskip=.3cm,rightskip=.3cm]%
    {author in head/foot}%
    \usebeamerfont{author in head/foot}%
    \hfill\insertshortauthor
  \end{beamercolorbox}%
  \begin{beamercolorbox}%
    [wd=.5\paperwidth,ht=2.5ex,dp=1.125ex,leftskip=.3cm ,rightskip=.3cm]%
    {title in head/foot}%
    \usebeamerfont{title in head/foot}%
    \insertshorttitle\hfill\insertframenumber{}
  \end{beamercolorbox}%
}%

\setbeamertemplate{caption}{\raggedright\insertcaption\par}
\captionsetup[figure]{font=small,skip=0pt}
\begin{frame}
\titlepage
\end{frame} 

\begin{frame}
\frametitle{Gliederung}
\tableofcontents
\end{frame} 

\newcommand{\tmotiv}{Grundlagen}
\section{\tmotiv}
\begin{frame} 

 \end{frame}


\section{Geladene Teilchen im EM-Feld}
\begin{frame}
 \frametitle{Hamiltonfunktion}
 \pause
 Elektromagnetische Potenziale:
 \begin{align*}
  \dive\, \mathbf{ B} &= 0 \rightarrow \mathbf{B} = \rot\, \mathbf{A}\\
  \rot\,\mathbf{ E}  &= -\dot {\mathbf{B}} = -\rot\, \frac{\partial\mathbf{A}}{\partial t} \rightarrow \mathbf{E} = -\grad\,\phi - \frac{\partial\mathbf{A}}{\partial t×}
 \end{align*}
 
 Aus Euler-Lagrange-Gleichung für die Lorentzkraft: 
 \begin{align*}
   \mathbf{F} &= e(\mathbf{E} + \dot {q}\times \mathbf{B}) = \frac{\dx}{\dx t} \frac{\partial U}{\partial \dot {q}} - \frac{\partial U}{\partial q}\\
   U &= e(\phi - \dot{q}\cdot\mathbf{A})
 \end{align*}
 
\end{frame}

\begin{frame}
 \frametitle{Hamiltonfunktion}
 \pause
 Lagrangefunktion:
 \begin{align*}
  L = T-U = \frac12 m \dot{q}^2 + e\dot{q}\cdot\mathbf{A} - e\phi
 \end{align*}
 Legendretransformation:
 \begin{align*}
  \underbrace{\mathbf{p}}_{kan.} &= \underbrace{m\mathbf{v}}_{mech.} + e\mathbf{A}\\
  H &= \mathbf{p}\dot{q} - L = (m\mathbf{v}+e\mathbf{A})\cdot \mathbf{v} - \frac12 m \mathbf{v}^2 - e \mathbf{v}\cdot\mathbf{A} + e\phi \\
  &= \frac12m\mathbf{v}^2 + e\phi
 \end{align*}
\end{frame}


\subsection{Relativistik}
 \begin{frame}
    \frametitle{Relativistische Erweiterung}
    \pause
  Ohne Feld:
  \begin{align*}
    H = \sqrt{\mathbf{p}^2_{\text{mech}}c^2 + m^2c^4},    \qquad \mathbf{p}_{\text{mech}} = \gamma m\mathbf{v}
  \end{align*}
  Mit Feld:
  \begin{align*}
   H = \sqrt{(\mathbf{p}_{\text{kan}}-e\mathbf{A})^2c^2 + m^2c^4}+e\phi
  \end{align*}
  Relativistische Lagrangefunktion:
  \begin{align*}
   L = -\frac{mc^2}{\gamma} + e \mathbf{A}\cdot\mathbf{v}-e\phi \neq T-U
  \end{align*}

 \end{frame}


\subsection{Transformation ins mitbewegete System}
\begin{frame}
\frametitle{Transformation ins mitbewegete System}
\pause
Neue Koordinaten $(Q_1,Q_2,Q_3) = (x,y,s)$ bzgl. Sollbahn $r_0(s)$:
\begin{align*}
 \mathbf{r}(s) = \rho\mathbf{e}_x(s) + x\mathbf{e}_x(s) + y\mathbf{e}_y(s) \qquad \rho = \text{Krümmungsradius}
\end{align*}
bild malen

\end{frame}

\begin{frame}
 \frametitle{Transformation ins mitbewegete System}
 \pause
 Frenet-Gleichungen (Torsion $\tau=0$):
 \begin{align*}
  \frac{\dx \mathbf{e}_s}{\dx s} = -\frac{1}{\rho}\mathbf{e}_x;\quad\frac{\dx \mathbf{e}_x}{\dx s} = \frac{1}{\rho}\mathbf{e}_s;\quad\frac{\dx \mathbf{e}_y}{\dx s} = 0
 \end{align*}
 Erzeugende der kanonischen Transformation:
 \begin{align*}
  F = F_3(\mathbf{p},\mathbf{Q}) + \mathbf{q}\cdot\mathbf{p}, \qquad F_3 = -\mathbf{r}(s)\cdot\mathbf{p}
 \end{align*}
 Neue Impulse $(P_x, P_y,P_s)$:
 \begin{align*}
  P_x &= \mathbf{p}\cdot\mathbf{e}_x; \hspace{1.45cm} P_y = \mathbf{p}\cdot\mathbf{e}_y;\\
  P_s &= \left(\rho\frac{1}{\rho}\mathbf{e}_s+x\frac{1}{\rho}\mathbf{e}_s\right)\mathbf{p} = \left(1+\frac{x}{\rho}\right)\mathbf{p}\mathbf{e}_s
 \end{align*}



\end{frame}


\subsection{Beispiele}
\begin{frame}
\frametitle{senkrechte Magnetfelder}
 \pause
 \begin{itemize}
  \visible<2->{\item Enthält dynamische Informationen (Wechselwirkungen)}
  \visible<3->{\item Beschreibt Übergang ähnlich Streuung von Startzustand $i$ zu Endzustand $f$}
  \visible<4->{\item Betragsquadrat $|M|^2$ kann als Wahrscheinlichkeit für Reaktion betrachtet werden}
 \end{itemize}
  \visible<5->{Ein erster Ausdruck:
  \begin{align*}
   M = \langle Kl\nu\,|\mathcal{H}|\,D\rangle
  \end{align*}}
\end{frame}

\begin{frame}
 \frametitle{kleine Winkel zur Sollbahn}
\end{frame}

\begin{frame}
 \frametitle{kleine Impulsabweichungen zum Sollimpuls}
\end{frame}

\section{Transformation auf Wirkungs-Winkel-Variable}
\subsection{Bedeutng der Wirkungs-Winkel-Variablen}
\begin{frame}
 \frametitle{Bedeutende Größen}
\end{frame}

\subsection{Beispiele}
\begin{frame}
 \frametitle{Beispiel: Gradientenfehler}
\end{frame}

\begin{frame}
 \frametitle{Beispiel: Sextupol}
\end{frame}

\section{Resonanzen}
\begin{frame}
 \frametitle{Beispiel: Sextupol}
\end{frame}

\begin{frame}
 \frametitle{Verhalten in Resonanznähe}
\end{frame}

\begin{frame}
 \frametitle{Fixpunkte}
\end{frame}

\section{Ausblick}
\begin{frame}
 \frametitle{Oktupol}
\end{frame}

\begin{frame}
 \frametitle{Kopplung}
\end{frame}

\begin{frame}
 \frametitle{Gegenwart vieler Nichtlinearitäten}
\end{frame}




\end{document}
=======

\documentclass[hyperref={pdfpagelabels=false}]{beamer}
% Die Hyperref Option hyperref={pdfpagelabels=false} verhindert die Warnung:
% Package hyperref Warning: Option `pdfpagelabels' is turned off
% (hyperref)                because \thepage is undefined. 
% Hyperref stopped early 
%
% \input{longheader.tex}
% \input{global.tex}

\usepackage{lmodern}
% Das Paket lmodern erspart die folgenden Warnungen:
% LaTeX Font Warning: Font shape `OT1/cmss/m/n' in size <4> not available
% (Font)              size <5> substituted on input line 22.
% LaTeX Font Warning: Size substitutions with differences
% (Font)              up to 1.0pt have occurred.
%

% % % % % % % % % % % % % % % % % % % % % % % % % % % % % % % % % % % % % % % % % % % %
\usepackage{siunitx}
\sisetup{load-configurations=abbreviations}
\sisetup{
	%locale=DE,
	seperr=true,                    % Fehler anzeigen
	tightpm,                        % Abstand zwischen Fehler verringern
	tophrase={{\text{ bis }}},
	fraction=nice,
	per-mode=fraction,
	free-standing-units=true,
	space-before-unit=true,
	use-xspace=true,
	group-separator={{\text{~}}},
	list-final-separator={{\text{ und }}}
}
\usepackage{natbib}
\usepackage[labelformat=empty]{caption}
\usepackage{movie15}
\usepackage{xcolor,colortbl}
\usepackage{slashed}
\usepackage{amsfonts}
\usepackage{amssymb}
\usepackage{amscd}
\usepackage{amstext}
\usepackage{amsthm}
\usepackage[ngerman,german]{babel, varioref}
\usepackage[T1]{fontenc}
\usepackage[utf8]{inputenc}
\usepackage{xfrac}
\usepackage{booktabs}

% % % % % % % % % % % % % % % % % % % % % % % % % % % % % % % % % % % % % % % % % % % % % % % % %
% Wenn \titel{\ldots} \author{\ldots} erst nach \begin{document} kommen,
% kommt folgende Warnung:
% Package hyperref Warning: Option `pdfauthor' has already been used,
% (hyperref) ... 
% Daher steht es hier vor \begin{document}

\title[Hamiltonformalismus]{Hamiltonformalismus in der Beschleunigerphysik}  
\institute{Beschleunigerphysik I \& II\\
Technische Universit\"at Dortmund}
\author{Dimitrios Skodras \and Sonja Bartkowski} 
\date{11.06.2015} 

% zusaetzlich ist das usepackage{beamerthemeshadow} eingebunden 
\usepackage{beamerthemeshadow}
\usepackage{amsmath}


%  \beamersetuncovermixins{\opaqueness<1>{25}}{\opaqueness<2->{15}}
%  sorgt dafuer das die Elemente die erst noch (zukuenftig) kommen 
%  nur schwach angedeutet erscheinen 
\beamersetuncovermixins{\opaqueness<1>{25}}{\opaqueness<2->{15}}
% klappt auch bei Tabellen, wenn teTeX verwendet wird\ldots

\beamertemplatenavigationsymbolsempty

\begin{document}

\setbeamertemplate{footline}
{%
  \leavevmode%
 \begin{beamercolorbox}%
    [wd=.5\paperwidth,ht=2.5ex,dp=1.125ex,leftskip=.3cm,rightskip=.3cm]%
    {author in head/foot}%
    \usebeamerfont{author in head/foot}%
    \hfill\insertshortauthor
  \end{beamercolorbox}%
  \begin{beamercolorbox}%
    [wd=.5\paperwidth,ht=2.5ex,dp=1.125ex,leftskip=.3cm ,rightskip=.3cm]%
    {title in head/foot}%
    \usebeamerfont{title in head/foot}%
    \insertshorttitle\hfill\insertframenumber{}
  \end{beamercolorbox}%
}%

\setbeamertemplate{caption}{\raggedright\insertcaption\par}
\captionsetup[figure]{font=small,skip=0pt}
\begin{frame}
\titlepage
\end{frame} 

\begin{frame}
\frametitle{Gliederung}
\tableofcontents
\end{frame} 


\section{Grundlagen}
 \begin{frame}
   \frametitle{Sonjas Teil}
   \pause
 \end{frame}
 

\section{Hamilton-Funktion für geladenen Teilchen im EM-Feld}
\begin{frame}
\frametitle{Dimitris Teil}
\tableofcontents[currentsection]
\end{frame}
\subsection{Gibt es hier Unterkapitel}
 

\section{Transformation auf Wirkungs-Winkel-Variable}
\begin{frame}
\frametitle{Sonjas Teil}
\tableofcontents[currentsection]
\end{frame}

\section{Kanonische Störungsrechnung}
\begin{frame}
\tableofcontents[currentsection]
\end{frame}
\subsection{Sextupol}
\begin{frame}
\frametitle{Sextupol - Sonjas Teil}
\end{frame}
\subsection*{Rest - Dimitris Teil?}
\begin{frame}
\frametitle{Rest - Dimitris Teil?}
 \framesubtitle{Motivation}
\end{frame}
\end{document}
>>>>>>> fb1dc7e1283ceaffe2a0f14f4c81cdfe63ef3826
